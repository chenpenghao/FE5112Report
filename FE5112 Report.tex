\documentclass[12pt]{article}
\usepackage[margin=1in]{geometry}
\usepackage[utf8]{inputenc}
\usepackage{amsmath}
\usepackage{amssymb}
\usepackage{listings}
\usepackage{mathtools}
\DeclarePairedDelimiter\ceil{\lceil}{\rceil}
\DeclarePairedDelimiter\floor{\lfloor}{\rfloor}
\newcommand{\N}{\mathbb{N}}
\renewcommand\thesubsubsection{\alph{subsubsection}}
\setlength{\parskip}{1em}
\setlength\parindent{0pt}
\renewcommand\arraystretch{1.5}
\usepackage{courier} %% Sets font for listing as Courier.
\usepackage{listings}
\usepackage[dvipsnames]{xcolor}
\definecolor{almond}{RGB}{246, 243, 199}
\lstset{
tabsize = 4, %% set tab space width
showstringspaces = false, %% prevent space marking in strings, string is defined as the text that is generally printed directly to the console
numbers = left, %% display line numbers on the left
commentstyle = \color{ForestGreen}, %% set comment color
keywordstyle = \color{blue}, %% set keyword color
stringstyle = \color{red}, %% set string color
rulecolor = \color{black}, %% set frame color to avoid being affected by text color
basicstyle = \small \ttfamily , %% set listing font and size
breaklines = true, %% enable line breaking
numberstyle = \tiny,
backgroundcolor=\color{almond}
}

% Bibliography
\usepackage[style=apa, backend=biber]{biblatex}
\addbibresource{5112.bib}

\begin{document}

\begin{titlepage}
\centering
{
	\includegraphics[scale=0.08]{./media/0.png}\\
	\vfill
	{\Large \textbf{FE5112}\par}
	{\Large \textbf{Stochastic Calculus and Quantitative Methods}\par}
	{\Large Project Report\par}
	\vfill
	{\normalsize Chen Penghao (A0122017Y)\par}
	{\normalsize Chia Kai Jit (A0122017Y)\par}
	{\normalsize Joanna See (A0122017Y)\par}
	{\normalsize Li Yihao (A0122017Y)\par}
	{\normalsize Zhang Zequn (A0156936R)\par}
	\vfill
% Bottom of the page
	{ November 2020\par}
}
\end{titlepage}

\section{Introduction}
The project aims to obtain the price of an Asian American put option by using the Least-Squared Monte Carlo simulation method.  We also have explored various techniques to reduce the variance of the results produced and to increase the stablity of the results.

\section{Problem Statement}

The market is complete and arbitrage free. The risk free interest rate is 8\%. The price of a non-dividend-paying stock $S_t$ at time $t$ is governed by the following SDE under the risk neutral probability measure $\mathbb{Q}$:

$$
\begin{cases}
dS_t = 0.08\hspace{1mm}S_t\hspace{1mm}dt + \sigma(S_t) \cdot S_t\hspace{1mm}dW_t, t > 0 \\
S_0 = 100
\end{cases}
$$

where $W_t$ is a Brownian motion under $\mathbb{Q}$ and $\sigma(\cdot)$ is a function given as the following:

$$
\sigma(x) = 
\begin{cases}
0.25+ 0.02 \times \left ( 1.0 - \dfrac{x}{S_0} \right), \text{if }x\leq 100 \\
\max\left(0.001, 0.25-0.01 \times\left(\dfrac{x}{S_0} - 1\right) \right), \text{if }x>100
\end{cases}
$$

We are to price a one-year-non-exercise American Asian put option with the strike price $K=108$ and the maturity $T=2$ in years. The option is not exercisable in the first year. The option is exercisable once and only once in the second year. If the option is exercised on a day $n$, where $366 \leq n \leq 730$, the option buyer will receive a payout of $(K-A_n)^+$, where $A_n$ is given by:
$$A_n = \dfrac{1}{60} \sum_{i=n-60}^{n-1} S_{\frac{i}{365}}$$

We assume that there are 365 days in a year and all days are business days.

\newpage
\section{LSMC with no variance reduction}

\subsection{Settings}

The variables are set up as the folloging:

\begin{lstlisting}[language=Python]
T = 2                       # Duration in years
L = 730                     # Number of time intervals
M = 500                     # Number of asset paths
rf = 0.08                   # Annualized risk-free rate
S0 = 100.0                  # Beginning stock price
dt = T/L                    # Time discretization (dt = 1/365)
dayEx = 730                 # Maximum days before the option exercises
discf = math.exp(-rf * dt)  # Discount factor to backwardize by one day.
\end{lstlisting}


\subsection{The Algorithm}

Our algorithm follows the following step:

\begin{enumerate}
\item Generate $M$ sample paths.
\item Generate a table of stock price data $S_L$ for 730 time intervals with 500 asset price paths. $S(i,j)$ would refer to the stock price on the $i^\text{th}$ path on the $j^\text{th}$ day.
\item Calculate the option payoff on the final day, the $L^\text{th}$ day or day 730.
\item Starting with the $(L-1)^\text{th}$ day, we do a backward propagation for each simulated path of the asset. For day i from $L-1$ back to 366:
	\begin{enumerate}
	\item Skip all the paths where it is not in the money on the $i^\text{th}$ day.
	\item If some path is in the money, 
	\end{enumerate}
\item The prices on all the sample pathsa re discounted to day 0. We simply take the average of the paths to obtain the fair value of the option.
\end{enumerate}

\subsection{Analysis of results}

\subsection{Conclusion}

\section{Variance reduction methods applied to LSMC}

\subsection{Antithetic Method}

\subsection{Control Variate Method}

\subsection{Importance Sampling Method}

\section{Conclusion}

\section{Final Words}

The contributions of our group members are as of the following:
\begin{itemize}
\item Kai Jit: Variance reduction with Antithetic method.
\item Penghao: Base code for LSMC and analysis.
\item Zequn: Variance reduction with importance sampling; Visualization of price paths; Refactor code base.
\item Yihao: Variance reduction with control variate.
\item Joanna:
\end{itemize}

\printbibliography

\end{document}